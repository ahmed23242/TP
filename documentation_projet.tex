\documentclass[12pt,a4paper]{article}
\usepackage[utf8]{inputenc}
\usepackage[french]{babel}
\usepackage{graphicx}
\usepackage{hyperref}
\usepackage{xcolor}
\usepackage{listings}
\usepackage{geometry}
\usepackage{enumitem}
\usepackage{titlesec}
\usepackage{fancyhdr}

\geometry{margin=2.5cm}

\titleformat{\section}
{\normalfont\Large\bfseries\color{blue}}
{\thesection}{1em}{}

\titleformat{\subsection}
{\normalfont\large\bfseries\color{blue!70}}
{\thesubsection}{1em}{}

\pagestyle{fancy}
\fancyhf{}
\fancyhead[L]{Application de Signalement d'Incidents Urbains}
\fancyhead[R]{\thepage}
\renewcommand{\headrulewidth}{0.4pt}

\begin{document}


\begin{titlepage}
    \centering
    \vspace*{2cm}
    {\Huge\bfseries Application de Signalement d'Incidents Urbains\par}
    \vspace{1.5cm}
    {\Large Documentation Technique\par}
    \vspace{2cm}
    {\Huge\bfseries  Faite par: Ahmed Abd Dayme AhmedBouha \par}
    \vspace{2cm}
    {\Large\itshape Rapport de Projet\par}
     \vspace{2cm}
    {\Huge\bfseries   \par}
    \vspace{2cm}
    \vfill
    {\large \today\par}
\end{titlepage}

\tableofcontents
\newpage

\section{Introduction}

Ce document présente une description détaillée de l'application mobile "Urban Incidents" développée avec Flutter. Cette application permet aux utilisateurs de signaler des incidents urbains (accidents, dégradations, problèmes d'infrastructure, etc.) en temps réel, avec la possibilité de fonctionner même sans connexion Internet grâce à un système de synchronisation avancé.

\subsection{Objectif du Projet}

L'objectif principal de ce projet est de fournir aux citoyens un outil simple et efficace pour signaler des incidents urbains aux autorités compétentes. L'application permet de :

\begin{itemize}
    \item Signaler rapidement des incidents avec photos et notes vocales
    \item Localiser précisément les incidents sur une carte
    \item Suivre l'état de traitement des signalements
    \item Fonctionner hors ligne avec synchronisation automatique
    \item Consulter l'historique des incidents signalés
\end{itemize}

\section{Architecture du Projet}

\subsection{Structure Générale}

Le projet suit une architecture modulaire basée sur le modèle MVC (Modèle-Vue-Contrôleur) avec une séparation claire des responsabilités. La structure des dossiers est organisée comme suit :

\begin{itemize}
    \item \textbf{lib/} : Contient tout le code source de l'application
    \begin{itemize}
        \item \textbf{core/} : Composants fondamentaux et services partagés
        \begin{itemize}
            \item \textbf{database/} : Gestion de la base de données locale SQLite
            \item \textbf{network/} : Services de communication avec l'API et gestion de la connectivité
            \item \textbf{services/} : Services généraux (permissions, navigation, etc.)
        \end{itemize}
        \item \textbf{features/} : Fonctionnalités organisées par domaine
        \begin{itemize}
            \item \textbf{auth/} : Authentification et gestion des utilisateurs
            \item \textbf{incidents/} : Gestion complète des incidents
        \end{itemize}
        \item \textbf{main.dart} : Point d'entrée de l'application
    \end{itemize}
    \item \textbf{assets/} : Ressources statiques (images, icônes, etc.)
    \item \textbf{backend/} : Code du serveur backend (API REST)
\end{itemize}

\subsection{Technologies Utilisées}

\begin{itemize}
    \item \textbf{Framework} : Flutter (Dart)
    \item \textbf{Gestion d'État} : GetX
    \item \textbf{Base de Données Locale} : SQLite (via sqflite)
    \item \textbf{Requêtes HTTP} : Dio
    \item \textbf{Cartographie} : flutter\_map avec OpenStreetMap
    \item \textbf{Localisation} : Geolocator
    \item \textbf{Capture Média} : image\_picker, audio\_recorder
    \item \textbf{Stockage Sécurisé} : flutter\_secure\_storage
    \item \textbf{Backend} : API REST (Django)
\end{itemize}

\section{Fonctionnalités Principales}

\subsection{Authentification}

\subsubsection{Inscription}
L'application permet aux utilisateurs de créer un nouveau compte en fournissant :
\begin{itemize}
    \item Nom d'utilisateur
    \item Adresse e-mail
    \item Mot de passe
    \item Numéro de téléphone (optionnel)
\end{itemize}

\subsubsection{Connexion}
Plusieurs méthodes de connexion sont disponibles :
\begin{itemize}
    \item Connexion par e-mail/mot de passe
    \item Connexion par authentification biométrique (empreinte digitale ou reconnaissance faciale)
\end{itemize}

\subsubsection{Gestion des Tokens}
\begin{itemize}
    \item Utilisation de tokens JWT pour l'authentification
    \item Rafraîchissement automatique des tokens expirés
    \item Stockage sécurisé des tokens dans le secure storage
\end{itemize}

\subsection{Gestion des Incidents}

\subsubsection{Création d'Incidents}
Les utilisateurs peuvent signaler un incident avec les informations suivantes :
\begin{itemize}
    \item Titre et description détaillée
    \item Type d'incident (accident, dégradation, infrastructure, etc.)
    \item Photo prise directement depuis l'application
    \item Note vocale pour décrire l'incident
    \item Localisation automatique par GPS
    \item Médias supplémentaires (optionnel)
\end{itemize}

\subsubsection{Visualisation des Incidents}
\begin{itemize}
    \item Liste des incidents avec filtres (type, statut)
    \item Carte interactive montrant les incidents à proximité
    \item Vue détaillée de chaque incident
    \item Historique complet des incidents signalés par l'utilisateur
\end{itemize}

\subsubsection{Synchronisation des Données}
\begin{itemize}
    \item Stockage local des incidents avec statut "pending" en mode hors ligne
    \item Synchronisation automatique lorsque la connexion Internet est rétablie
    \item Fusion des données locales et distantes pour un affichage complet
    \item Synchronisation périodique (toutes les 5 minutes)
\end{itemize}

\subsection{Fonctionnalités Avancées}

\subsubsection{Capture Multimédia}
\begin{itemize}
    \item Prise de photos avec l'appareil photo du téléphone
    \item Enregistrement de notes vocales
    \item Compression et optimisation des médias avant envoi
\end{itemize}

\subsubsection{Géolocalisation et Cartographie}
\begin{itemize}
    \item Détection automatique de la position de l'utilisateur via le package Geolocator
    \item Affichage des incidents sur une carte interactive
    \item Calcul de la distance entre l'utilisateur et les incidents
\end{itemize}

\paragraph{Implémentation de la Carte}
L'application utilise le package \texttt{flutter\_map} pour l'affichage des cartes avec les caractéristiques suivantes :
\begin{itemize}
    \item Fournisseur de tuiles : OpenStreetMap (\texttt{https://\{s\}.tile.openstreetmap.org/\{z\}/\{x\}/\{y\}.png})
    \item Zoom initial adaptatif (13.0 à proximité de l'utilisateur, 2.0 par défaut)
    \item Marqueurs personnalisés pour les incidents avec code couleur selon le statut :
    \begin{itemize}
        \item Vert : incidents résolus
        \item Bleu : incidents en cours de traitement
        \item Orange : incidents en attente
        \item Rouge : incidents urgents ou autres
    \end{itemize}
    \item Interaction directe avec les marqueurs pour accéder aux détails des incidents
    \item Bouton flottant pour signaler un nouvel incident à la position actuelle
\end{itemize}

\paragraph{Fonctionnalités de la Carte}
\begin{itemize}
    \item Centrage automatique sur la position de l'utilisateur au démarrage
    \item Affichage en temps réel des incidents à proximité
    \item Navigation fluide avec contrôles de zoom et de déplacement
    \item Optimisation des performances avec chargement asynchrone des tuiles
    \item Support du mode hors ligne avec mise en cache des tuiles récemment consultées
\end{itemize}

\subsubsection{Mode Hors Ligne}
\begin{itemize}
    \item Fonctionnement complet sans connexion Internet
    \item Stockage local des données en attente de synchronisation
    \item Indicateurs visuels pour les incidents non synchronisés
\end{itemize}

\subsubsection{Statistiques}
\begin{itemize}
    \item Nombre total d'incidents signalés
    \item Répartition par type d'incident
    \item Statut des incidents (en attente, en cours, résolu)
\end{itemize}

\subsubsection{Internationalisation}
\begin{itemize}
    \item Support multilingue (français, anglais)
    \item Adaptation aux préférences régionales
    \item Utilisation du drapeau américain pour l'option de langue anglaise
\end{itemize}

\section{Composants Techniques}

\subsection{Base de Données Locale}

La base de données SQLite locale est gérée par la classe \texttt{DatabaseHelper} et comprend les tables suivantes :

\subsubsection{Table Users}
\begin{itemize}
    \item id : Identifiant unique
    \item email : Adresse e-mail (unique)
    \item name : Nom complet
    \item password : Mot de passe (stocké de manière sécurisée)
    \item phone : Numéro de téléphone
    \item role : Rôle de l'utilisateur
    \item token : Token d'authentification
    \item last\_login : Date de dernière connexion
\end{itemize}

\subsubsection{Table Incidents}
\begin{itemize}
    \item id : Identifiant unique
    \item title : Titre de l'incident
    \item description : Description détaillée
    \item photo\_path : Chemin local de la photo
    \item photo\_url : URL de la photo sur le serveur
    \item voice\_note\_path : Chemin local de la note vocale
    \item latitude : Coordonnée géographique
    \item longitude : Coordonnée géographique
    \item created\_at : Date de création
    \item status : État de traitement (pending, in\_progress, resolved)
    \item incident\_type : Type d'incident
    \item sync\_status : État de synchronisation (pending, synced)
    \item user\_id : Identifiant de l'utilisateur
    \item additional\_media : Médias supplémentaires (JSON)
\end{itemize}

\subsection{Communication Réseau}

\subsubsection{ApiClient}
La classe \texttt{ApiClient} gère toutes les communications avec le backend :
\begin{itemize}
    \item Utilisation de Dio pour les requêtes HTTP
    \item Gestion automatique des tokens d'authentification
    \item Rafraîchissement des tokens expirés
    \item Gestion des erreurs réseau
\end{itemize}

\subsubsection{ConnectivityService}
Le service \texttt{ConnectivityService} surveille l'état de la connexion Internet :
\begin{itemize}
    \item Détection des changements de connectivité
    \item Notification des autres services lors des changements
    \item Vérification de la disponibilité du backend
\end{itemize}

\subsection{Services Principaux}

\subsubsection{IncidentService}
Le service \texttt{IncidentService} est responsable de la gestion complète des incidents :
\begin{itemize}
    \item Création et stockage local des incidents
    \item Récupération des incidents depuis la base de données et le backend
    \item Capture des photos et enregistrement audio
    \item Obtention de la localisation
\end{itemize}

\subsubsection{SyncService}
Le service \texttt{SyncService} gère la synchronisation des données :
\begin{itemize}
    \item Synchronisation automatique lors de la reconnexion
    \item Synchronisation périodique
    \item Gestion des conflits de données
    \item Mise à jour des statuts de synchronisation
\end{itemize}

\subsubsection{AuthService}
Le service \texttt{AuthService} gère l'authentification des utilisateurs :
\begin{itemize}
    \item Inscription et connexion
    \item Gestion des tokens JWT
    \item Stockage sécurisé des informations d'authentification
    \item Vérification de l'état de connexion
\end{itemize}

\section{Interface Utilisateur}

\subsection{Écrans Principaux}

\subsubsection{Écran de Connexion (LoginScreen)}
Interface permettant à l'utilisateur de se connecter avec ses identifiants ou par biométrie.

\subsubsection{Écran d'Accueil (HomeScreen)}
Tableau de bord principal affichant :
\begin{itemize}
    \item Résumé des incidents récents
    \item Statistiques personnelles
    \item Accès rapide aux fonctionnalités principales
\end{itemize}

\subsubsection{Écran de Carte (MapScreen)}
Carte interactive basée sur OpenStreetMap montrant :
\begin{itemize}
    \item Position actuelle de l'utilisateur avec centrage automatique
    \item Incidents à proximité avec marqueurs colorés selon leur statut
    \item Affichage du titre de l'incident sur chaque marqueur
    \item Interaction tactile pour accéder aux détails d'un incident
    \item Bouton flottant pour signaler un nouvel incident à l'emplacement actuel
    \item Contrôles de zoom et de navigation fluide
\end{itemize}

\subsubsection{Écran de Création d'Incident (CreateIncidentScreen)}
Formulaire complet pour signaler un nouvel incident avec :
\begin{itemize}
    \item Champs de texte pour le titre et la description
    \item Sélection du type d'incident
    \item Boutons pour capturer photo et audio
    \item Affichage de la localisation actuelle
\end{itemize}

\subsubsection{Écran de Détails d'Incident (IncidentDetailsScreen)}
Vue détaillée d'un incident spécifique avec :
\begin{itemize}
    \item Toutes les informations de l'incident
    \item Affichage des médias (photos, audio)
    \item Statut actuel et historique des mises à jour
    \item Position sur une mini-carte
\end{itemize}

\subsubsection{Écran d'Historique (IncidentHistoryScreen)}
Liste complète des incidents signalés par l'utilisateur avec :
\begin{itemize}
    \item Filtres par type et statut
    \item Options de tri (date, type, statut)
    \item Indicateurs de synchronisation
\end{itemize}

\subsubsection{Écran des Incidents en Attente (PendingIncidentsScreen)}
Liste des incidents non encore synchronisés avec le serveur.

\section{Gestion des Données}

\subsection{Modèle de Données}

\subsubsection{Classe Incident}
Modèle principal pour les incidents avec les propriétés suivantes :
\begin{itemize}
    \item id : Identifiant unique
    \item title : Titre de l'incident
    \item description : Description détaillée
    \item photoPath : Chemin local de la photo
    \item photoUrl : URL de la photo sur le serveur
    \item voiceNotePath : Chemin local de la note vocale
    \item latitude, longitude : Coordonnées géographiques
    \item createdAt : Date de création
    \item status : État de traitement
    \item incidentType : Type d'incident
    \item syncStatus : État de synchronisation
    \item userId : Identifiant de l'utilisateur
    \item additionalMedia : Liste des médias supplémentaires
\end{itemize}

\subsection{Flux de Données}

\subsubsection{Création d'un Incident}
\begin{enumerate}
    \item L'utilisateur remplit le formulaire et soumet l'incident
    \item Le contrôleur appelle le service d'incidents
    \item L'incident est créé avec le statut "pending"
    \item Les données sont stockées dans la base de données locale
    \item Si une connexion Internet est disponible, la synchronisation est tentée immédiatement
\end{enumerate}

\subsubsection{Synchronisation}
\begin{enumerate}
    \item Le service de synchronisation détecte une connexion Internet
    \item Les incidents avec statut "pending" sont récupérés de la base de données
    \item Chaque incident est envoyé au serveur via l'API
    \item En cas de succès, le statut est mis à jour à "synced"
    \item Les incidents du serveur sont récupérés et fusionnés avec les données locales
\end{enumerate}

\subsubsection{Affichage des Incidents}
\begin{enumerate}
    \item Le contrôleur demande les incidents au service
    \item Le service récupère les incidents locaux depuis la base de données
    \item Si une connexion est disponible, les incidents distants sont également récupérés
    \item Les données sont fusionnées pour éviter les doublons
    \item La liste complète est retournée au contrôleur pour affichage
\end{enumerate}

\section{Problèmes Résolus}

\subsection{Disparition des Incidents}
Un problème majeur a été résolu concernant la disparition des incidents de l'historique après leur création. La solution implémentée :
\begin{itemize}
    \item Les incidents sont systématiquement stockés localement avec le statut "pending"
    \item Ils restent visibles dans l'historique indépendamment de l'état de connectivité
    \item La synchronisation est gérée en arrière-plan sans affecter l'affichage
\end{itemize}

\subsection{Synchronisation Incomplète}
Un autre problème concernait l'affichage incomplet des incidents. La solution :
\begin{itemize}
    \item L'application récupère désormais tous les incidents du backend
    \item Ces incidents sont fusionnés avec les incidents locaux
    \item Un système de déduplication évite les doublons dans l'affichage
    \item L'historique complet est maintenant visible
\end{itemize}

\subsection{Préférences Régionales}
Conformément aux préférences de l'utilisateur, le drapeau américain (US) est maintenant utilisé pour l'option de langue anglaise au lieu du drapeau britannique (UK).

\section{Sécurité}

\subsection{Authentification}
\begin{itemize}
    \item Utilisation de tokens JWT pour l'authentification
    \item Stockage sécurisé des tokens via flutter\_secure\_storage
    \item Rafraîchissement automatique des tokens expirés
    \item Déconnexion automatique en cas d'échec de rafraîchissement
\end{itemize}

\subsection{Protection des Données}
\begin{itemize}
    \item Chiffrement de la base de données locale
    \item Transmission des données via HTTPS
    \item Validation des données côté client et serveur
    \item Gestion sécurisée des informations sensibles
\end{itemize}



\section{Interface d'Administration Web}

En plus de l'application mobile, le projet comprend une interface d'administration web complète permettant aux administrateurs de gérer l'ensemble du système.

\subsection{Tableau de Bord Administrateur}

L'interface d'administration est centrée autour d'un tableau de bord complet qui offre une vue d'ensemble du système.

\subsubsection{Statistiques et Indicateurs}

Le tableau de bord présente plusieurs indicateurs clés de performance :
\begin{itemize}
    \item Nombre total d'incidents signalés
    \item Incidents récents (7 derniers jours)
    \item Nombre total d'utilisateurs
    \item Nombre d'utilisateurs actifs
    \item Répartition des incidents par statut (avec visualisation par barres de progression)
    \item Répartition des incidents par type (avec visualisation par barres de progression)
\end{itemize}

\subsubsection{Carte Interactive des Incidents}

Une carte interactive affiche tous les incidents géolocalisés avec des marqueurs de couleur différente selon leur statut :
\begin{itemize}
    \item Rouge : incidents urgents
    \item Orange : incidents en attente
    \item Bleu : incidents en cours de traitement
    \item Vert : incidents résolus
\end{itemize}

Les administrateurs peuvent cliquer sur les marqueurs pour accéder directement aux détails de l'incident.

\subsubsection{Listes Récapitulatives}

Le tableau de bord présente également :
\begin{itemize}
    \item Les 5 derniers utilisateurs inscrits avec le nombre d'incidents qu'ils ont signalés
    \item Les 5 incidents les plus récents avec leur statut et type
\end{itemize}

\subsection{Gestion des Incidents}

L'interface d'administration permet une gestion complète des incidents :

\subsubsection{Liste des Incidents}
\begin{itemize}
    \item Affichage de tous les incidents avec filtres par statut, type et recherche textuelle
    \item Possibilité de filtrer les incidents par utilisateur
    \item Actions rapides : voir détails, modifier, résoudre, supprimer
    \item Mise à jour du statut directement depuis la liste
\end{itemize}

\subsubsection{Détails et Édition des Incidents}
\begin{itemize}
    \item Vue détaillée de chaque incident avec toutes les informations
    \item Formulaire d'édition pour modifier les propriétés de l'incident
    \item Possibilité d'ajouter des médias supplémentaires
    \item Visualisation de la position sur une carte
    \item Gestion des fichiers (photos, notes vocales)
\end{itemize}

\subsection{Gestion des Utilisateurs}

\subsubsection{Liste des Utilisateurs}
\begin{itemize}
    \item Affichage de tous les utilisateurs avec filtres et recherche
    \item Statistiques par utilisateur (nombre d'incidents, date d'inscription)
    \item Actions rapides : voir détails, voir incidents, supprimer
\end{itemize}

\subsubsection{Détails et Édition des Utilisateurs}
\begin{itemize}
    \item Vue détaillée du profil utilisateur
    \item Modification des informations du compte
    \item Gestion des droits d'accès
    \item Possibilité de désactiver un compte
\end{itemize}

\subsection{Sécurité de l'Administration}

\begin{itemize}
    \item Accès restreint aux utilisateurs avec le statut administrateur
    \item Authentification obligatoire pour accéder au tableau de bord
    \item Journalisation des actions administratives
    \item Protection contre les attaques CSRF
    \item Validation des données côté serveur
\end{itemize}

\subsection{Technologies Utilisées}

\begin{itemize}
    \item Backend : Django (Python)
    \item Interface utilisateur : Bootstrap 5 avec thème SB Admin 2
    \item Visualisation des données : JavaScript natif
    \item Cartographie : Google Maps API ou Leaflet
    \item Authentification : Django Authentication System
\end{itemize}

\section{Conclusion}

L'application "Urban Incidents" offre une solution complète et robuste pour le signalement d'incidents urbains. Ses principales forces sont :

\begin{itemize}
    \item Fonctionnement hors ligne avec synchronisation intelligente
    \item Interface utilisateur intuitive et réactive
    \item Capture multimédia (photos, audio) intégrée
    \item Géolocalisation précise des incidents
    \item Interface d'administration web complète pour la gestion du système
    \item Architecture modulaire facilitant la maintenance et l'évolution
\end{itemize}

Cette application répond efficacement aux besoins des utilisateurs tout en garantissant la fiabilité et la sécurité des données. Sa conception permet une évolution future pour intégrer de nouvelles fonctionnalités et s'adapter aux besoins changeants des utilisateurs.

\end{document}
